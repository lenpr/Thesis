% --	Definition von globalen Parametern, die im gesamten Dokument verwendet werden
\usepackage[utf8]{inputenc}			% muss schon hier aktiviert werden für Umlaute

\newcommand{\titel}{Fast User-Guided Mesh-Simplification with Topology Control}
\newcommand{\untertitel}{TopStoc}
\newcommand{\art}{Diploma Thesis}
\newcommand{\fachgebiet}{Department of Computer Graphics\\ TU-Berlin} 
\newcommand{\autor}{Lukas N.P. Egger}
\newcommand{\mail}{\href{mailto:mail@lnpe.at}{mail@lnpe.at}}
\newcommand{\studienbereichA}{Industrial Engineering}
\newcommand{\studienbereichB}{\& Philosophy}
\newcommand{\matrikelnummer}{22 48 48}
\newcommand{\gutachter}{Prof. Dr. Marc Alexa}
\newcommand{\zweitgutachter}{Prof. Dr. Bernd Bickel}
\newcommand{\betreuer}{Dipl. Inf. Ronald Richter}
\newcommand{\jahr}{2012}
\newcommand{\abgabe}{$31^{st}$ of July, 2012}
\newcommand{\AutorZ}[1]{\textsc{#1}}

% Autorenname in small caps
\newcommand{\Autor}[1]{\AutorZ{\citesuthor{#1}}}

% Abkürzungen
\newcommand{\bs}{$\backslash$}

% Befehle zur semantischen Auszeichnung von Text
\newcommand{\newterm}[1]{\textbf{#1}}
\newcommand{\techterm}[1]{\textit{#1}}
\newcommand{\code}[1]{\texttt{#1}}
\newcommand{\file}[1]{\texttt{#1}}

% Befehle um links hochgestellte Zeichen benutzen zu können, besser als \sideset
\newcommand\Prefix[3]{\vphantom{#3}#1#2#3}

% \Item schreibt eine Equation Nummer ans Ende der Zeile, die Zeile muss aber mit $ anfangen 
\def\Item$#1${\item $\displaystyle#1$
\hfill\refstepcounter{equation}(\theequation)}
