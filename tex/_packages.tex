% Anpassung des Seitenlayouts, siehe Seitenstil.tex
\usepackage[
	automark,			% Kapitelangaben in Kopfzeile automatisch erstellen
	headsepline,			% Trennlinie unter Kopfzeile
	ilines				% Trennlinie linksbündig ausrichten
]{scrpage2}

% Wichtig für korrekte Zitierweise (round, square, etc.)
\usepackage[square]{natbib}

% Quellenangaben in integrierter amerik. Version
%\bibpunct{(}{)}{;}{a}{}{,~}

% Landessprache, verwendet globale Option siehe \documentclass
\usepackage{babel}
\usepackage{varioref} 		% eigenartige Fehler ohne, stellt Option anderen Modulen zur Verfügung

% Umlaute, erlaubt automatische Trennung von Worten mit Umlauten
%\usepackage[utf8]{inputenc}	% ist in _meta, da dort schon Umlaute
\usepackage[T1]{fontenc}
\usepackage{lmodern} 			% "schöneres" ß
\usepackage{textcomp} 		% Euro-Zeichen etc.

% Grafiken, einbinden von Grafiken [draft oder final]
\usepackage[dvips,final]{graphicx}
\graphicspath{{pics/}} 		% dort liegen die Bilder des Dokuments

% Befehle aus AMSTeX für mathematische Symbole z.B. \boldsymbol \mathbb
\usepackage{amsmath,amsfonts}

% Index-Ausgabe \printindex
\usepackage{makeidx}

% Einfache Definition der Zeilenabstände und Seitenränder etc.
\usepackage{setspace}
\usepackage{geometry}

% ein Bild wird in den laufenden Text eingebunden, Text läuft [l] oder [r] vorbei
\usepackage{floatflt}

% Lange URLs umbrechen etc.
\usepackage{url}

% Zum fortlaufenden Durchnummerieren der Fußnoten
\usepackage{chngcntr}

% Symbolverzeichnis, beruht auf MakeIndex, die Definitionen sind ausgegliedert
%\usepackage[toc, acronym]{glossaries}


% Zum Einbinden von Programmcode
\usepackage{listings}
\usepackage{xcolor} 
\definecolor{hellgelb}{rgb}{1,1,0.9}
\definecolor{colKeys}{rgb}{0,0,1}
\definecolor{colIdentifier}{rgb}{0,0,0}
\definecolor{colComments}{rgb}{1,0,0}
\definecolor{colString}{rgb}{0,0.5,0}
\lstset{%
    float=hbp,%
    basicstyle=\texttt\small, %
    identifierstyle=\color{colIdentifier}, %
    keywordstyle=\color{colKeys}, %
    stringstyle=\color{colString}, %
    commentstyle=\color{colComments}, %
    columns=flexible, %
    tabsize=2, %
    frame=single, %
    extendedchars=true, %
    showspaces=false, %
    showstringspaces=false, %
    numbers=left, %
    numberstyle=\tiny, %
    breaklines=true, %
    backgroundcolor=\color{hellgelb}, %
    breakautoindent=true, %
    inputencoding=utf8x
    %captionpos=b%
}

% für lange Tabellen
\usepackage{longtable}
\usepackage{array}
\usepackage{ragged2e}
\usepackage{lscape}

% Spaltendefinition rechtsbündig mit definierter Breite
\newcolumntype{w}[1]{>{\raggedleft\hspace{0pt}}p{#1}}

% Formatierung von Listen ändern
\usepackage{paralist}
% Standardeinstellungen:
\setdefaultleftmargin{2.5em}{1.5em}{1.87em}{1.7em}{1em}{1em}

% PDF-Optionen
\usepackage[
	bookmarks,
	bookmarksopen=true,
	pdftitle={\titel},
	pdfauthor={\autor},
	pdfcreator={\autor},
	pdfsubject={\titel},
	pdfkeywords={\titel},
	colorlinks=true,
	% für die Druckversion können die Farben ausgeschaltet werden:
	linkcolor=blue, 			% einfache interne Verknüpfungen
	anchorcolor=black,		% Ankertext
	citecolor=blue, 			% Verweise auf Literaturverzeichniseinträge im Text
	filecolor=magenta, 		% Verknüpfungen, die lokale Dateien öffnen
	menucolor=red, 		% Acrobat-Menüpunkte
	urlcolor=cyan, 
	%linkcolor=black, 		
	%anchorcolor=black,	
	%citecolor=black, 		
	%filecolor=black, 
	%menucolor=black, 
	%urlcolor=black, 
	backref,
	%pagebackref,
	plainpages=false,		% zur korrekten Erstellung der Bookmarks
	pdfpagelabels,		% -||-
	hypertexnames=false,	% -||-
	linktocpage 			% Seitenzahlen anstatt Text im Inhaltsverzeichnis verlinken
]{hyperref}

% \Item schreibt eine Equation Nummer ans Ende der Zeile, die Zeile muss aber mit $ anfangen 
\def\Item$#1${\item $\displaystyle#1$
\hfill\refstepcounter{equation}(\theequation)}

% Packages um Pseudocode zu erzeugen und zu nutzen
%\usepackage{algorithm}
%\usepackage{algorithmic}
