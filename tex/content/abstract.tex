%\vspace*{1ex}
\section*{Abstract}
\label{abstract}

This thesis focuses on the user-guided simplification of geometric models, i.e. the decimation of 3D embedded triangle meshes.
Within this area, research spends its attention mostly on deterministic and fully automated algorithms.
The goal of this work however is not to add a new part to the standard ensemble, but instead look into the gains that can be achieved by user interaction combined with non-deterministic approaches.
Thus not only the possibility for artistic control changes, but also the benchmarks for various aspects of the problem, e.g. speed has to be gauged with the requirements for fluid interactive work in mind.
Additionally we show how techniques derived from the field of topological persistence can be used to effectively measure and control the topology of a mesh.\\
Apart from the obligatory discussion of results there is also a brief historical outline of mesh simplification, a primer to the mathematical skill-set necessary to tackle the emerging phenomena, as well as an outlook on further research questions.\\[1ex]

\section*{Zusammenfassung}
\label{zusammenfassung}

Diese Arbeit widmet sich der nutzergestützten Vereinfachung von geometrischen Modellen.
Innerhalb dieses Bereiches liegt der Fokus gewöhnlich auf deterministischen und vollautomatischen Algorithmen.
Ziel dieser Arbeit ist es, nicht dem Kanon an Standardansätzen einen neuen beizufügen, sondern stattdessen die Vorteile von Benutzerinteraktion in Kombination mit nicht-deterministischen Ansätzen zu beleuchten. 
Dadurch verschiebt sich nicht nur die artistischen Kontrollmöglichkeiten, sondern auch der Maßstab, bspw. muss Geschwindigkeit fortan anhand direktem und flüssigen Arbeitens bewertet werden.
Zusätzlich zeigen wir, wie mit Techniken der topologischen Persistenz die Eigenschaften eines Modells ermittelt und verändert werden können.\\
Neben dem obligatorischen Teil Auswertungsteil, gibt es auch einen kurzen geschichtlichen Abriss der Entwicklung dieses Spezialgebietes, als auch eine Einführung in die Teile der Topologie, welche zur Beschreibung der auftretenden Phänomene benutzt werden, sowie einen Ausblick auf weiterführende Fragestellungen. 
