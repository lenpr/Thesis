\chapter{User-Guided Mesh Simplification}
\label{topstoc0}

\begin{flushright}
\textit{``The original motivation to do research [that led to technological discoveries]\\was to expand the range of possibilities for storytellers.''}\\
-- Ed Catmull {[}at the 73rd Scientific and Technical Academy Awards, 2001{]}
\end{flushright}


%\newpage
%\vspace*{1ex}
\section{TopStoc Algorithm}
\label{topstoc1}

Lorem Ipsum.

\subsection{Reception}
\label{topstoc11}

Lorem Ipsum.

\subsection{Adaptations}
\label{topstoc12}

Lorem Ipsum.

\subsubsection{Early List Collapse}
\label{topstoc121}

Lorem Ipsum.

\subsubsection{Adaptive Floodfill}
\label{topstoc122}

Lorem Ipsum.

\subsubsection{Saving Cluster Data}
\label{topstoc123}


%\newpage
%\vspace*{1ex}
\section{Simplifying Topology}
\label{topstoc2}

Our goal is to provide the user with clear and unambiguous information, so that he can easily decide whether to dismiss or keep a topological feature.
It took the entire chapter \ref{math0}, to lay down the mathematical foundations and introduce the algorithms to do so, yet it is very important not to underestimate the role of presentation.
Therefore we want to talk about how to visualize our results.\\
Lorem Ipsum.


%\newpage
%\vspace*{1ex}
\section{Drawing on Meshes}
\label{topstoc3}

Lorem Ipsum.

\subsection{Explicit Control}
\label{topstoc31}

\subsection{Sketchy Regions}
\label{topstoc32}


%\newpage
%\vspace*{1ex}
\section{Context Setting}
\label{topstoc4}

Lorem Ipsum.

\subsection{Point of View}
\label{topstoc41}

\subsection{Textures}
\label{topstoc42}

\subsection{Bounding Boxes}
\label{topstoc43}

\subsection{Materials}
\label{topstoc44}

\newpage
\section{Meassurements}
\label{topstoc5}

\subsection{Hausdorff Distance}
\label{topstoc51}

In the case of 1D-signals and images, many distortion measurements have been studied.
They range from simple analytic methods such as the mean square error to very elaborate techniques based on the characteristics of human perception \citep[cf.][]{Winkler2001}.
Despite the constantly growing usage of 3D models, the distortion measurements for such data seem to be less thorough.
Still the predominant approach to compare two meshes is the \textit{Hausdorff distance}\footnote{ First introduced in the book \textit{Grundzüge der Mengenlehre}, published in 1914 and named after the German mathematician \textit{Felix Hausdorff} (1868 – 1942), one of the founders of modern topology. Technically the Hausdorff distance it is not a real distance function since it lacks symmetry. It is better to think of it in terms of set containment, for more information and examples see Appendix \ref{appendix5}.}, defined as:
\begin{equation} \label{eq:hausdorff}
\hspace*{0.5cm} d_{\mathrm H}(\mathcal{M},\mathcal{M}') = \sup_{p \in \mathcal{M}} \inf_{p' \in \mathcal{M}'} d(p, p')
\end{equation}
respectively the symmetric version:
\begin{equation}
d_{\mathrm H_{sym}}(\mathcal{M},\mathcal{M}') = \max \{\sup_{p \in \mathcal{M}} \inf_{p' \in \mathcal{M}'} d(p, p')~, \sup_{p' \in \mathcal{M}'} \inf_{p \in \mathcal{M}} d(p', p)\}
\end{equation}
Where the distance $d(p, p')$ between two points $p$ and $p'$ on the surfaces of the mesh $\mathcal{M}$ and it's simplified version $\mathcal{M}'$ is defined as: $d(p, p') = || p-p' ||$ with $||.||$ denoting the usual Euclidean norm.
We will refer to $d_{\mathrm H}(\mathcal{M},\mathcal{M}')$ as the forward distance, and to $d_{\mathrm H}(\mathcal{M}',\mathcal{M})$ as the backward distance.\\
The distance between any point $p$ belonging to $\mathcal{M}$ and $\mathcal{M}'$ can be computed analytically, since it can be reduced to the minimum of the distances between p and all the triangles $\mathcal{F}_{\mathcal{M}'}$.
It is worth noting that $p$ might not be a vertex of the list $\mathcal{V}_{\mathcal{M}}$ but any point on the surface.
Hence, although straightforward, the algorithm becomes too complex if implemented naively\footnote{ In fact, for each sample point $p$ it would be necessary to calculate the distance to all triangles $\mathcal{F}_{\mathcal{M}'}$ in order to find the minimum.
This leads to a complexity $\mathcal{O}(|p| \cdot |\mathcal{F}_{\mathcal{M}'}|)$, where $|p|$ is the number of sample points taken on $\mathcal{M}$ and $|\mathcal{F}_{\mathcal{M}'}|$ is the number of triangles.}.
To achieve reasonable running times we followed the implementation strategy described in the paper \textit{``Mesh: Measuring errors between surfaces using the hausdorff distance''}, where a cell partitioning strategy is used to greatly reduces the number of point-triangle distance evaluations.
Each triangles is sampled via a regular grid according to a length criterion, thus avoiding uniform sampling \citep[for more details see][especially chapter 3]{Aspert2002}.

still hausdorff important here

for POV simplification makes no sense

two approaches, we focus on visual error

\subsection{Multiview}
\label{topstoc52}
